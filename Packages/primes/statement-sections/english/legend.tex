$Mazen$ is facing a challenging problem and he requires your assistance.

To begin, let's establish the definition of prime factorization:

We denote the function $P(x)$ as the product of the prime factors of a given number $x$, arranged in ascending order.

For instance:
$P(10)$ = $2 \times 5$, and
$P(12)$ = $2 \times 2 \times 3$.

Now, let's introduce the function $S(x)$, which represents the sum of the prime factors of a given number $x$, also arranged in ascending order.

For instance:
$S(10)$ = $2 + 5$ = $7$, and
$S(12)$ = $2 + 2 + 3$ = $7$.

The problem involves having $n$ distinct prime factors and a value $m$. Your task is to count the number of different numbers $x$ for which $S(x)$ = $m$ and we can form $x$ using some or all $n$ distinct prime factors and we can use any of $n$ factors any number of times.

Output the answer mod $10^9 + 7$.